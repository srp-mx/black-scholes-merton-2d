\documentclass[main.tex]{subfiles}
\usepackage{util/estilo}

\begin{document}

Una opción sobre una acción (o simplemente \textit{opción})
\parencite{optionsBasics} \parencite[10-11]{tesis-bsm} es un contrato que
permite a quien lo posee, pero no lo obliga, a comprar (\textit{opción de
compra}) o vender (\textit{opción de venta}) más adelante una acción a un
precio acordado (\textit{precio de ejercicio} \parencite[12]{tesis-bsm}), con
una fecha límite de expiración (o de vencimiento) \parencite[12]{tesis-bsm}.

Lógicamente si en un futuro la acción vale más, entonces el comprador sale
beneficiado pues la puede adquirir a un precio menor al valor en el mercado.
Por el contrario si la acción vale menos, entonces la opción es inútil pues el
comprador puede simplemente comprarla a su valor actual en el mercado, o bien,
puede no comprarla. \parencite[11]{tesis-bsm}

Una opción sobre una acción puede ser de \textit{estilo europeo} o
\textit{americano}. En el estilo americano la opción puede ser ejercida en
cualquier momento hasta la fecha de expiración, mientras que en el estilo
europeo la opción, de ser ejercida, debe serlo exactamente en la fecha de
expiración. \parencite[10-13]{tesis-bsm} \parencite{investopedia-options}

Llamamos un \textit{activo subyacente} (o simplemente \textit{activo}) al
instrumento financiero en el cual un \textit{derivado} se basa. Una opción
sobre una acción es un derivado cuyo activo subyacente es la acción misma, y la
acción representa una fracción de propiedad sobre una empresa.
\parencite[13]{tesis-bsm} \parencite{optionsBasics}
\footnote{\textit{underlying security}}

Según \parencite{bsm73}, el valor de una opción europea sin costos de
transacción sobre un único activo, bajo el modelo de Black-Scholes-Merton
(BSM), es dado por la ecuación diferencial parcial de segundo orden:
\begin{equation}
    \dparc{V}{t} + \frac{1}{2}\sigma^2S^2\dparc{^2V}{S^2} + rS\dparc{V}{S} - rV = 0
    \hspace{2mm} ; \hspace{2mm}
    0 \leq t \leq T
\label{eq:bsm1}
\end{equation}
donde $V(S,t)$ es el valor de la opción con precio del activo $S=S(t)$ en
tiempo $t$ en años, $T$ el tiempo de expiración, $r$ es la tasa de interés
compuesto continuamente sin riesgo anualizada, $\sigma$ es la desviación
estándar del rendimiento de la acción anualizado (el promedio de su rendimiento
anual) y se asume que el riesgo financiero de la opción es nulo,
usualmente mediante una estrategia de \textit{cobertura dinámica}.
\parencite{bsm73} \parencite{bs-formula} \parencite[20-21]{tesis-bsm}

Para plantear bien el problema, es necesario condicionar la EDP, lo cual se
hace al decidir si nuestra opción es de compra o de venta. \parencite{bsm73}
Empíricamente, el modelo BSM no es tan adecuado para opciones de venta, pero sí
(bajo reservaciones) para opciones de compra. \parencite{bs-puts-calls}
\parencite{bs-formula} Por lo tanto, nos centraremos en el valor de las
opciones de compra, el cual se expresa en términos de un valor final y valores
en la frontera, donde $V=C$ y $K$ es el precio de ejercicio:
\begin{equation}
    C(0,t) = 0
\label{eq:frontera_call1}
\end{equation}
\begin{equation}
    \lim_{S\to\infty} C(S,t) = S - K
\label{eq:frontera_call2}
\end{equation}
\begin{equation}
    C(S,T) = \max\cjto{S-K,0}
\label{eq:final_call}
\end{equation}

Si $\rho$ es el coeficiente de correlación entre ambos activos, entonces la
ecuación BSM para dos activos, asumiendo cero dividendos como en
\parencite{bsm73}, se puede reducir de la ecuación 1 de \parencite{unam-bsm2d}
en:
\begin{multline}
    \dparc{V}{t} + rS_1\dparc{V}{S_1} + rS_2\dparc{V}{S_2} + \frac{1}{2}\sigma_1^2S_1^2\dparc{^2V}{S^2_1} + \frac{1}{2}\sigma_2^2S_2^2\dparc{^2V}{S^2_2} \\
    + \rho \sigma_1 \sigma_2 S_1 S_2 \dparc{^2V}{S_1\partial S_2} - rV = 0
\label{eq:bsm2}
\end{multline}

Para las condiciones de frontera y finales, si $\varphi_K(S_1,S_2)$ es la
manera en que se valúa el precio conjunto de ambos activos, i.e. la función de
pago asociada a la opción (ya considerando el precio de ejercicio), entonces
\begin{equation}
    C(0,S_2,t) = \varphi_K(0, S_2)
\label{eq:frontera2_call1}
\end{equation}
\begin{equation}
    C(S_1,0,t) = \varphi_K(S_1, 0)
\label{eq:frontera2_call2}
\end{equation}
\begin{equation}
    \lim_{S_1+S_2\to\infty} C(S_1,S_2,t) = \frac{\varphi_K(S_1,S_2)}{S_1+S_2}
\label{eq:frontera2_call3}
\end{equation}
\begin{equation}
    C(S_1,S_2,T) = \max\cjto{\varphi_K(S_1,S_2),0}
\label{eq:final2_call}
\end{equation}
% TODO: Revisa si la condición final hace sentido, no se si más bien debería
% ser algo así como lim_{S_1->inf} C(S_1,S_2,t) = C(S_2,t), etc. 

\end{document}

