\documentclass[main.tex]{subfiles}
\usepackage{util/estilo}

%% TODO

\begin{document}

Sean
$$x_1 = \ln{S_1} \hspace{4mm} x_2 = \ln{S_2} \hspace{4mm} \tau = T - t$$
$$U(x_1,x_2,\tau) = V(S_1,S_2,t)$$

Por la regla de la cadena tenemos
\begin{multline}
    -\dparc{U}{\tau} + \frac{1}{2}\sigma_1^2\dparc{^2U}{x_1^2} + \frac{1}{2}\sigma_2^2\dparc{^2U}{x_2^2} + \rho\sigma_1\sigma_2\dparc{^2U}{x_1\partial x_2}\\
    + \pp{r-\frac{1}{2}\sigma_1^2}\dparc{U}{x_1} + \pp{r-\frac{1}{2}\sigma_2^2}\dparc{U}{x_2} - rU = 0
\label{eq:bsm2d_t1}
\end{multline}

Sea $W(x_1,x_2,\tau)=e^{r\tau}U(x_1,x_2,\tau)$, de donde $W_\tau = re^{r\tau}U
+ e^{r\tau}U_\tau$ y así sustituyendo $U_\tau$ despejado de lo anterior,
\begin{multline}
    -\dparc{W}{\tau} + \frac{1}{2}\sigma_1^2\dparc{^2W}{x_1^2} + \frac{1}{2}\sigma_2^2\dparc{^2W}{x_2^2} + \rho\sigma_1\sigma_2\dparc{^2W}{x_1\partial x_2}\\
    + \pp{r-\frac{1}{2}\sigma_1^2}\dparc{W}{x_1} + \pp{r-\frac{1}{2}\sigma_2^2}\dparc{W}{x_2} = 0
\label{eq:bsm2d_t2}
\end{multline}

Supongamos $a_1,a_2\in\mathbb{R}$ constantes. Definamos
$$Y(x_1,x_2,\tau) = e^{-a_1x_1-a_2x_2}W(x_1,x_2,\tau) = e^{-\flecha{a}\cdot \flecha{x}}W$$

Observemos que
\begin{equation}
    \dparc{Y}{x_i}e^{\flecha{a}\cdot \flecha{x}} = \dparc{W}{x_i}-a_iW
\label{eq:y_xi}
\end{equation}
\begin{equation}
    \dparc{^2Y}{x_i^2}e^{\flecha{a}\cdot\flecha{x}} = \dparc{^2W}{x_i^2}-2a_i\dparc{W}{x_i} + a_i^2W
\label{eq:y_xi2}
\end{equation}
\begin{equation}
    \dparc{^2Y}{x_1\partial x_2}e^{\flecha{a}\cdot\flecha{x}} = \dparc{^2W}{x_1\partial x_2} - a_1\dparc{W}{x_2} - a_2\dparc{W}{x_1} + a_1a_2W
\label{eq:y_x1x2}
\end{equation}
\begin{equation}
    \dparc{Y}{\tau}e^{\flecha{a}\cdot\flecha{x}} = \dparc{W}{\tau}
\label{eq:y_tau}
\end{equation}

Notemos que tras sustituir, los coeficientes de las primeras derivadas desaparecen
si escogemos $\flecha{a}$ tal que
\begin{equation}
    \begin{pmatrix}
        \sigma_1^2 & \rho\sigma_1\sigma_2 \\[1.5ex]
        \rho\sigma_1\sigma_2 & \sigma_2^2
    \end{pmatrix}\flecha{a}
    =
    \begin{pmatrix}
        r-\frac{1}{2}\sigma_1^2 \\[1.5ex]
        r-\frac{1}{2}\sigma_2^2
    \end{pmatrix}
\label{eq:desp_a}
\end{equation}

De donde
\begin{equation}
    \flecha{a}
    =
    \frac{1}{\sigma_1^2\sigma_2^2\pp{1-\rho^2}}
    \begin{pmatrix}
        \sigma_2^2 & -\rho\sigma_1\sigma_2 \\[1.5ex]
        -\rho\sigma_1\sigma_2 & \sigma_1^2
    \end{pmatrix}
    \begin{pmatrix}
        r-\frac{1}{2}\sigma_1^2 \\[1.5ex]
        r-\frac{1}{2}\sigma_2^2
    \end{pmatrix}
\label{eq:desp_a_fin}
\end{equation}

Esto nos da ahora a nuestra ecuación diferencial como
\begin{equation}
    -\dparc{Y}{\tau} + \frac{1}{2}\sigma_1^2\dparc{^2Y}{x_1^2} + \frac{1}{2}\sigma_2^2\dparc{^2Y}{x_2^2} + \rho\sigma_1\sigma_2\dparc{^2Y}{x_1\partial x_2} + \lambda Y = 0
\label{eq:edp_y}
\end{equation}
con $\lambda = \sigma_1^2a_1^2/2 + \sigma_2^2a_2^2/2 + \rho\sigma_1\sigma_2a_1a_2 - (r-\sigma_1^2/2)a_1 - (r-\sigma_2^2/2)a_2$,
el cual podemos reducir con $Z(x_1,x_2,\tau)=e^{-\lambda\tau}Y(x_1,x_2,\tau)$:
\begin{equation}
    -\dparc{Z}{\tau} + \frac{1}{2}\sigma_1^2\dparc{^2Z}{x_1^2} + \frac{1}{2}\sigma_2^2\dparc{^2Z}{x_2^2} + \rho\sigma_1\sigma_2\dparc{^2Z}{x_1\partial x_2} = 0
\label{eq:edp_z}
\end{equation}

Podemos hacer un cambio de coordenadas para eliminar el término mixto. Notemos
que dada la matriz
$$
A = \frac{1}{2}
    \begin{pmatrix}
        \sigma_1^2 & \rho\sigma_1\sigma_2 \\[1.5ex]
        \rho\sigma_1\sigma_2 & \sigma_2^2
    \end{pmatrix}
$$

Entonces su forma cuadrática en los operadores de derivada parcial con respecto
a $x_1$ y $x_2$ es:
\begin{multline}
    \begin{pmatrix}
        \partial_{x_1} & \partial_{x_2}
    \end{pmatrix}
    A
    \begin{pmatrix}
        \partial_{x_1} \\
        \partial_{x_2}
    \end{pmatrix}Z \\
    =
\frac{1}{2}\sigma_1^2\dparc{^2Z}{x_1^2} + \frac{1}{2}\sigma_2^2\dparc{^2Z}{x_2^2} + \rho\sigma_1\sigma_2\dparc{^2Z}{x_1\partial x_2}
\label{eq:qform_z}
\end{multline}

Así, podemos diagonalizar $A$ para obtener las coordenadas que nos eliminan el
término mixto al ser ortogonal. Sea $Q$ tal que $QAQ^T = \textrm{diag}(\mu_1,
\mu_2)$ con $\mu_1\neq \mu_2$. Así
$\flecha{y}=Q\flecha{x}$. De esto,
\begin{equation}
    -\dparc{Z}{\tau} + \mu_1\dparc{^2Z}{y_1^2} + \mu_2\dparc{^2Z}{y_2^2} = 0
\label{eq:edp_zy}
\end{equation}

Como caso especial para evitar degeneraciones, si $\rho=0$, entonces tomaremos
$\flecha{x}=\flecha{y}$, $\mu_1=\sigma_1^2/2$ y $\mu_2=\sigma_2^2/2$ pues ya
está en la forma que queremos, sin importar si $\mu_1=\mu_2$.

Finalmente, reescalamos las coordenadas para no tener coeficientes. Con
$\xi_1 = y_1/\sqrt{2\mu_1}$ y $\xi_2 = y_2/\sqrt{2\mu_2}$ obtenemos
\begin{equation}
    \dparc{Z}{\tau} = \dparc{^2Z}{\xi_1^2} + \dparc{^2Z}{\xi_2^2}
\label{eq:calor_bsm}
\end{equation}

Recuperando el valor de la opción por sustituciones,
\begin{equation}
    V(S_1,S_2,t) = e^{(\lambda-r)(T-t) + \flecha{a}\cdot\pp{\ln{S_1},\ln{S_2}}}Z(\xi_1,\xi_2,T-t)
\label{eq:calor_bsm_inv}
\end{equation}
recordando los valores de $\lambda$, $\flecha{a}$ y $\xi_i$, los cuales
provienen de transformaciones simples a la solución de sistemas lineales
$2\times 2$.

Si $E_{Mi}$ denota al $i$-ésimo eigenvalor de $M$, $\flecha{E}_{Mi}$ el
$i$-ésimo eigenvector de $M$ y $\alpha = e^{(\lambda-r)(T-t) +
\flecha{a}\cdot\pp{\ln{S_1},\ln{S_2}}}$, tenemos
\begin{equation}
    V(S_1,S_2,t) =
    \alpha Z\pp{
        \frac{\flecha{E}_{A1}^T}{\sqrt{2E_{A1}}}
        \begin{pmatrix}
            \ln{S_1} \\
            \ln{S_2}
        \end{pmatrix}
    ,\frac{\flecha{E}_{A2}^T}{\sqrt{2E_{A2}}}
        \begin{pmatrix}
            \ln{S_1} \\
            \ln{S_2}
        \end{pmatrix}
    ,T-t}
\label{eq:calor_bsm_inv2}
\end{equation}

Resolviendo directamente los eigenvalores y eigenvectores de $A$, tenemos
\begin{equation}
 E_{A1} = \frac{1}{2}\pp{\sigma_1^2 + \sigma_2^2 - \sqrt{(\sigma_1^2 - \sigma_2^2)^2 + 4\rho^2\sigma_1^2\sigma_2^2}}
\label{eq:eigval_a1}
\end{equation}
\begin{equation}
 E_{A2} = \frac{1}{2}\pp{\sigma_1^2 + \sigma_2^2 + \sqrt{(\sigma_1^2 - \sigma_2^2)^2 + 4\rho^2\sigma_1^2\sigma_2^2}}
\label{eq:eigval_a2}
\end{equation}
\begin{equation}
 \flecha{E}_{A1}^T = \begin{pmatrix} 1 & -\frac{\sigma_1^2-E_{A1}}{\rho\sigma_1\sigma_2} \end{pmatrix}
     \hspace{4mm}
 \flecha{E}_{A2}^T = \begin{pmatrix} 1 & -\frac{\sigma_1^2-E_{A2}}{\rho\sigma_1\sigma_2} \end{pmatrix}
\label{eq:eigvec_a1a2}
\end{equation}

Como $-1 \leq \rho \leq 1$ por ser correlación, los eigenvalores serán todos
reales no-negativos, por lo que $A$ es semidefinido positivo. Notemos que
$E_{A1} = E_{A2} \iff \rho^2=1$, lo cual es un caso degenerado.

Así, la transformación de $S_i$ hacia $\xi_i$ en $0<|\rho|<1$ es
\begin{equation}
    \begin{pmatrix}
        \xi_1 \\
        \xi_2
    \end{pmatrix}
    =
    \begin{pmatrix}
        \frac{1}{\sqrt{2E_{A1}}} & \frac{E_{A1} - \sigma_1^2}{\rho\sigma_1\sigma_2\sqrt{2E_{A1}}} \\
        \frac{1}{\sqrt{2E_{A2}}} & \frac{E_{A2}-\sigma_1^2}{\rho\sigma_1\sigma_2\sqrt{2E_{A2}}}
    \end{pmatrix}
    \begin{pmatrix}
        \ln{S_1} \\[1ex]
        \ln{S_2}
    \end{pmatrix}
\label{eq:s_to_xi_neq0}
\end{equation}

Y por otro lado si $\rho = 0$, notando que en finanzas $\sigma_i > 0$,
\begin{equation}
    \begin{pmatrix}
        \xi_1 \\
        \xi_2
    \end{pmatrix}
    =
    \begin{pmatrix}
        \sigma_1^{-1}\ln{S_1} \\[1ex]
        \sigma_2^{-1}\ln{S_2}
    \end{pmatrix}
\label{eq:s_to_xi_eq0}
\end{equation}

Así, si $0<|\rho|<1$, $\textrm{exp}[\flecha{v}] = (e^{v_1},
e^{v_2})^T$ y $\nu=\rho\sigma_1\sigma_2$, invirtiendo
\begin{equation}
    \begin{pmatrix}
        S_1 \\
        S_2
    \end{pmatrix}
    =
    \textrm{exp}\ppsq{
        \frac{2\nu\sqrt{E_{A1}E_{A2}}}{E_{A2}-E_{A1}}
        \begin{pmatrix}
            \frac{E_{A2}-\sigma_1^2}{\nu\sqrt{2E_{A2}}} & \frac{\sigma_1^2-E_{A1}}{\nu\sqrt{2E_{A1}}} \\
            -\frac{1}{\sqrt{2E_{A2}}} & \frac{1}{\sqrt{2E_{A1}}}
        \end{pmatrix}
        \begin{pmatrix}
            \xi_1 \\[1ex]
            \xi_2
        \end{pmatrix}
    }
\label{eq:xi_to_s_neq0}
\end{equation}

Y si $\rho = 0$,
\begin{equation}
    \begin{pmatrix}
        S_1 \\
        S_2
    \end{pmatrix}
    =
    \begin{pmatrix}
        e^{\sigma_1\xi_1} \\
        e^{\sigma_2\xi_2}
    \end{pmatrix}
\label{eq:xi_to_s_eq0}
\end{equation}

%% TODO: Revisar si esto no es del todo correcto
Finalmente queda por considerar el caso $|\rho|=1$, pero notemos que de ser
este el caso entonces el valor de ambos activos es linealmente dependiente por
tener correlación con magnitud 1, por lo que el problema se puede reducir a la
ecuación del calor 1D en la dirección del único eigenvector sin despreciar la
dependencia de función de pago en ambas coordenadas originales. No
contemplaremos este caso por simplicidad.

%% TODO: Revisa condiciones de frontera, no son realmente correctas
Trazando las sustituciones a las condiciones del sistema, tenemos
\begin{equation}
    \lim_{\xi_1\to -\infty} Z(\xi_1,\xi_2,\tau) = \varphi_K(0, S_2)
\label{eq:fronteraz_call1}
\end{equation}
\begin{equation}
    \lim_{\xi_2 \to -\infty} Z(\xi_1,\xi_2,\tau) = \varphi_K(S_1, 0)
\label{eq:fronteraz_call2}
\end{equation}
\begin{equation}
    \lim_{\xi_1+\xi_2\to\infty} Z(\xi_1,\xi_2,\tau) = \frac{\varphi_K(S_1,S_2)}{S_1+S_2}
\label{eq:fronteraz_call3}
\end{equation}
\begin{equation}
    Z(\xi_1,\xi_2,0) = \max\cjto{\varphi_K(S_1,S_2),0}
\label{eq:finalz_call}
\end{equation}

\end{document}

