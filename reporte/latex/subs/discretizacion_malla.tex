\documentclass[main.tex]{subfiles}
\usepackage{util/estilo}

\begin{document}

Trabajamos en $z$\textit{-coordenadas} con tiempo $\tau$, sobre las cuales
resolveremos numéricamente la ecuación de calor para obtener un estimado de
$u(z_1,z_2,\tau)$.

El desafío de la discretización de la malla yace en la aplicación de las
condiciones de frontera, pues por un lado ahora estas ocurren en los límites
mientras cada $z_i\to\pm\infty$, y por el otro lado dependen de las coordenadas
$S_i$ y $y_i$, las cuales no necesariamente están alineadas a los ejes en
$z$-coordenadas.

Para resolver estos problemas, proponemos dar un dominio cuadrado grande, que
se escale con respecto a la máxima desviación estándar (i.e.
$\sigma_1,\sigma_2$) y el valor máximo de $\sqrt{\tau}$, en el cual pretendemos
que donde sea que se encuentren los nodos externos se les asignará la condición
de frontera de Dirichlet correspondiente a su orientación hacia el infinito y
si $S_1+S_2$ excede un umbral relativo al máximo valor de $S_1+S_2$. Mientras
más amplio sea el dominio, entonces será mejor la aproximación.

Esta decisión nos inducirá sesgo en nuestro modelo, pues no debería en estos
puntos de la frontera mantenerse siempre el mismo valor por más que sea muy
grande y en un inicio se aproxime bien el valor en el límite. Quedará por
verse cómo se compara con algún $\varphi_K$ para el cual se conozca alguna
solución exacta.

\end{document}

