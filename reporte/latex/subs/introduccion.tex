\documentclass[main.tex]{subfiles}
\usepackage{util/estilo}

\begin{document}

\PARstart{L}a ecuación de Black-Scholes-Merton (BSM) \parencite{bsm73} es uno
de los modelos más influyentes en su ámbito \parencite{bs-approach}
\parencite{bs-formula}, siendo ampliamente usada como un punto de
partida para métodos modernos de valuación de títulos opcionales sobre acciones
en el mercado bursátil, el cual es un problema central de las finanzas
cuantitativas. \parencite[20-21]{tesis-bsm}

Es bien conocido, incluso utilizado en la publicación original de Black y
Scholes, que la ecuación BSM se puede reducir en la ecuación de calor en una
dimensión espacial y una temporal \parencite{bsm73}
\parencite{bs-heat-transform}, lo cual la amena a tratamientos analíticos y
numéricos efectivos. \parencite{bs-lecture-notes} \parencite{bs-heat-transform}

Mientras que la ecuación BSM en una dimensión nos da el valor de una opción
sobre una acción al estilo europeo de un activo \parencite{bsm73}
\parencite{bs-presentation} \parencite{bs-lecture-notes}, existen opciones
multi-activo que no subyacen su valor en un único activo, como es el caso de
canastas de mercancías, opciones para dos acciones, o contratos que involucran
múltiples tipos de cambio. \parencite{unam-bsm2d}

Este reporte se enfoca en comparar métodos numéricos para resolver la ecuación
BSM en dos activos subyacentes, mediante su reducción a la ecuación de calor en
dos dimensiones espaciales y una temporal, haciendo énfasis en la precisión,
la estabilidad numérica y el tiempo de cómputo de los esquemas numéricos
implementados en GPGPU.

\end{document}

