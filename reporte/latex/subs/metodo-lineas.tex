\documentclass[main.tex]{subfiles}
\usepackage{util/estilo}

\begin{document}

Discretizando únicamente el espacio en la ecuación de calor, obtenemos el lado
derecho de \ref{eq:ftcs0}, el cual se puede despejar a algo similar a
\ref{eq:ftcs1} con $S_\mu=\frac{\alpha}{\Delta\mu^2}$ y donde
$S_{xy}=-2S_x-2S_y$ ya que el $1$ provendría del lado izquierdo. Esto nos deja
con el esquema de problema de valor inicial ($t=0$)
\begin{equation}
    \left.\deriv{\textrm{\textbf{u}}}{t}\right\rvert_{t=n\Delta t} = f_n = M'U^n
\label{eq:lin-scheme1}
\end{equation}

Con $M'$ una matriz con diagonal principal $S_{xy}$, diagonal $1$ y $-1$
como $S_x$ y diagonal $w$ y $-w$ como $S_y$, asumiendo que en $U$ vectorizamos
la malla 2D como hicimos en la sección \ref{sec:crank-nicholson}. Separando los
nodos interiores y de frontera obtenemos
\begin{equation}
    \underbrace{f_{n(I)}}_{\deriv{\textrm{\textbf{u}}}{t}} = \underbrace{M'_{(II)}}_{M}\underbrace{U_{(I)}^n}_{\textrm{\textbf{u}}^n} + \underbrace{M'_{(IF)}U_{(F)}^n}_{\textrm{\textbf{b}}}
\label{eq:lin-scheme2}
\end{equation}

Como Euler explícito es $U^{n+1} = U^n + \Delta t f_n$, notemos que así nuestros
$S_\mu$ en $f_n$ se multiplican por $\Delta t$, coincidiendo con $S_x$ y $S_y$
en la ecuación \ref{eq:ftcs1}, y sumando el término $U^n$ se agrega en la $i$-ésima
entrada $U_i^n$, por lo que ahora $S_{xy}$ también coincide pues podemos sumarle $1$
para contabilizar la contribución de esta suma. Por lo tanto, el esquema resultante
de aplicar Euler explícito es exactamente el mismo que el de la sección \ref{sec:ftcs}.

Por otro lado, aplicando un esquema RK4, observando que $b$ es la única parte
dependiente del tiempo, pues contiene los nodos de frontera, tenemos
\begin{equation}
    u^{n+1} = u^n + \frac{\Delta t}{6}\pp{\kappa_1 + 2\kappa_2 + 2\kappa_3 + \kappa_4}
\label{eq:rk4-lin1}
\end{equation}
$$ \kappa_1 = Mu^n + b^n \hspace{2mm};\hspace{2mm}
\kappa_2 = M\pp{u^n + \frac{\Delta t}{2}\kappa_1} + b^{n+\frac{1}{2}} $$
$$ \kappa_3 = M\pp{u^n + \frac{\Delta t}{2}\kappa_2} + b^{n+\frac{1}{2}} \hspace{2mm};\hspace{2mm}
\kappa_4 = M\pp{u^n + \Delta t \kappa_3} + b^{n+1} $$

Notemos que como $M$ depende de $\Delta x$ y $\Delta y$, entonces la
estabilidad de nuestro esquema y su \textit{stiffness} (rigidez) dependerá de
la discretización espacial. Por el método de discos de Gershgorin, tenemos por
cada fila un círculo con centro $c$ y radio $R$ tal que
\begin{equation}
    c = -2\alpha\pp{\Delta x^{-2} + \Delta y^{-2}}
    \hspace{2mm};\hspace{2mm}
    R = 2\alpha\pp{\Delta x^{-2} + \Delta y^{-2}}
\label{eq:gersh-lin}
\end{equation}

Así $\lambda\in[-4\alpha\pp{\Delta x^{-2}+\Delta y^{-2}},0]$ eigenvalor de $M$
cualquiera. Esto nos dice que los eigenvectores son negativos y
$|\lambda_{\max{}}| \leq 4\alpha\pp{\Delta x^{-2} + \Delta y^{-2}}$, lo que nos
indica que mientras más fina sea la malla, mas rígido es el sistema.

Esto no nos da el radio de rigidez $|\lambda_{\max{}}|/|\lambda_{\min{}}|$,
pero sí nos da idea de este comportamiento. En la ejecución de nuestro
simulador daremos tal radio con el método de la potencia.

\end{document}


