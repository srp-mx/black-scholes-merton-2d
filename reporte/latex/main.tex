\documentclass[9pt,conference,letterpaper]{IEEEtran}
\usepackage{util/estilo}

% Title and authors
\title{
    \rule{\linewidth}{2pt}
    \Title{Soluciones Numéricas a la Ecuación de Calor Aplicadas a la Ecuación
    de Black-Scholes-Merton 2D en CUDA Python} \\[-1ex]
    \rule{\linewidth}{2pt}
}

\author{
    \IEEEauthorblockN{Santiago Romero}
    \IEEEauthorblockA{Estudiante de Ciencias de la Computación\\
    Facultad de Ciencias UNAM, Ciudad de México\\
    \email{santiago.rp@ciencias.unam.mx}}
}

% Document starts
\begin{document}

\maketitle

\thispagestyle{plain}
\pagestyle{plain}

\begin{abstract}
    %% TODO
\end{abstract}

\begin{IEEEkeywords}
    %% TODO
\end{IEEEkeywords}

\section{Introducción}\label{sec:intro}
\subfile{subs/introduccion.tex}

\section{Preliminares}\label{sec:preliminares}
\subfile{subs/preliminares.tex}

\section{Planteamiento Matemático y Numérico} \label{sec:planteamiento}

\subsection{BSM Como Ecuación de Calor 2D}
\subfile{subs/bsm2d.tex}

\subsection{Métodos Numéricos}

\subsubsection{Discretización de la malla}\label{sec:malla}
\subfile{subs/discretizacion_malla.tex}
\vspace*{1em}

\subsubsection{Esquema Forward-Time Central-Space (FTCS)}\label{sec:ftcs}
\subfile{subs/ftcs.tex}
\vspace*{1em}

\subsubsection{Esquema Crank-Nicholson}\label{sec:crank-nicholson}
\subfile{subs/crank-nicholson.tex}
\vspace*{1em}

\subsubsection{Esquema de Método de Líneas}\label{sec:metodo-lineas}
\subfile{subs/metodo-lineas.tex}

\section{Metodología}\label{sec:metodologia}

\subsection{Implementación}

\section{Resultados y Discusión}\label{sec:resultados}
\subfile{subs/resultados.tex}

\section{Conclusión}
\subfile{subs/conclusion.tex}\label{sec:conclusion}

\printbibliography

\end{document}
